El programa en Python se encarga de tomar 100 imágenes con la cámara
web de la computadora y guardarlas numerándolas de la 0.png a la 99.png.
Estas imágenes son las que se utilizan en el proceso de calibración,
aunque solamente 25 son necesarias. Las imágenes se toman cada 10
frames de video, con el costo de computación, esto equivale a aproximadamente
una imagen por segundo.

El código de calibración de la cámara esta basado en el ejemplo de
OpenCV, este código recibe de entrada un XML de configuración que
contiene las dimensiones del tablero y el amanto de cada casilla.
En este caso se utilizo un tablero de 9 x 6 y tamaño del cuadro 25
mm. Ademas de lo anterior, en el XML de configuración se especifica
la fuente de las imágenes para hacer la calibración; en este proyecto
se utilizaron imágenes generadas anteriormente con un script de Python,
por lo que hay que especificar en un XML separado el nombre de todas
las imágenes utilizadas.

La función \textquotedblleft findChessboardCorners\textquotedblright{}
encuentra los cruces del tablero en la imagen y retorna una matriz
de los puntos de imagen. Esta matriz posteriormente se le pasa como
parámetro a calibrateCamera para encontrar la matriz de parámetros
de la cámara.

La función \textquotedblleft calibrateCamera\textquotedblright{} es
la que se encarga de refinar los parámetros intrínsecos y extrínsecos
de la cámara, con la ayuda del patrón de calibración. Esta función
trata de minimizar el error de proyección y al final retorna dicho
valor. El error obtenido durante las pruebas de la calibración esta
alrededor de los 2mm

Al finalizar la ejecución se guardan los resultados en un XML con
la función \textquotedblleft saveCameraParams\textquotedblright ,
el cual es la entrada de la siguiente etapa del proyecto. Ademas,
la aplicación muestra las imágenes de entrada sin distorsión para
poder apreciar si el efecto es el esperado.
